% Copyright 2008 by Christian Feuersaenger
%
% This file may be distributed and/or modified
%
% 1. under the LaTeX Project Public License and/or
% 2. under the GNU Free Documentation License.
%
% See the file doc/generic/pgf/licenses/LICENSE for more details.


\section{Floating Point Unit Library}
\label{pgfmath-floatunit}
\label{section-library-fpu}

{\noindent {\emph{by Christian Feuersänger}}}

\begingroup
\pgfqkeys{/pgf/number format}{sci}
\pgfkeys{/pgf/fpu}

\begin{pgflibrary}{fpu}
    The floating point unit (fpu) allows the full data range of scientific
    computing for use in \pgfname. Its core is the \pgfname\ math routines for
    mantissa operations, leading to a reasonable trade-of between speed and
    accuracy. It does not require any third-party packages or external
    programs.
\end{pgflibrary}


\subsection{Overview}

The fpu provides a replacement set of math commands which can be installed in
isolated placed to achieve large data ranges at reasonable accuracy. It
provides at least%
    \footnote{To be more precise, the FPU's exponent is currently a 32-bit
    integer. That means it supports a significantly larger data range than an
    IEEE double precision number -- but if a future \TeX\ version may provide
    low-level access to doubles, this may change.}%
the IEEE double precision data range, $\pgfmathprintnumber{-1e+324}, \dotsc,
\pgfmathprintnumber{+1e324}$. The absolute smallest number bigger than zero is
$\pgfmathprintnumber{1e-324}$. The FPU's relative precision is at least
$\pgfmathprintnumber{1e-4}$ although operations like addition have a relative
precision of $\pgfmathprintnumber{1e-6}$.

Note that the library has not really been tested together with any drawing
operations. It should be used to work with arbitrary input data which is then
transformed somehow into \pgfname\ precision. This, in turn, can be processed
by \pgfname.


\subsection{Usage}

\begin{key}{/pgf/fpu=\marg{boolean} (default true)}
    This key installs or uninstalls the FPU. The installation exchanges any
    routines of the standard math parser with those of the FPU: |\pgfmathadd|
    will be replaced with |\pgfmathfloatadd| and so on. Furthermore, any number
    will be parsed with |\pgfmathfloatparsenumber|.
    %
\begin{codeexample}[preamble={\usepgflibrary{fpu}}]
\pgfkeys{/pgf/fpu}
\pgfmathparse{1+1}\pgfmathresult
\end{codeexample}
    %
    \noindent The FPU uses a low-level number representation consisting of
    flags, mantissa and exponent%
        \footnote{Users should \emph{always} use high
        level routines to manipulate floating point numbers as the format may
        change in a future release.}.%
    To avoid unnecessary format conversions, |\pgfmathresult| will usually
    contain such a cryptic number. Depending on the context, the result may
    need to be converted into something which is suitable for \pgfname\
    processing (like coordinates) or may need to be typeset. The FPU provides
    such methods as well.

%--------------------------------------------------
% TODOsp: codeexamples: Why is this example commented?
% \begin{codeexample}[preamble={\usepgflibrary{fpu}}]
% \begin{tikzpicture}
%     \fill[red,fpu,/pgf/fpu/scale results=1e-10] (*1.234e10,*1e10) -- (*2e10,*2e10);
% \end{tikzpicture}
% \end{codeexample}
%--------------------------------------------------

    Use |fpu=false| to deactivate the FPU. This will restore any change. Please
    note that this is not necessary if the FPU is used inside of a \TeX\ group
    -- it will be deactivated afterwards anyway.

    It does not hurt to call |fpu=true| or |fpu=false| multiple times.

    Please note that if the |fixedpointarithmetic| library of \pgfname\ will
    be activated after the FPU, the FPU will be deactivated automatically.
\end{key}

\begin{key}{/pgf/fpu/output format=\mchoice{float,sci,fixed} (initially float)}
    This key allows to change the number format in which the FPU assigns
    |\pgfmathresult|.

    The predefined choice |float| uses the low-level format used by the FPU.
    This is useful for further processing inside of any library.
    %
\begin{codeexample}[preamble={\usepgflibrary{fpu}}]
\pgfkeys{/pgf/fpu,/pgf/fpu/output format=float}
\pgfmathparse{exp(50)*42}\pgfmathresult
\end{codeexample}

    The choice |sci| returns numbers in the format
    \meta{mantissa}|e|\meta{exponent}. It provides almost no computational
    overhead.
    %
\begin{codeexample}[preamble={\usepgflibrary{fpu}}]
\pgfkeys{/pgf/fpu,/pgf/fpu/output format=sci}
\pgfmathparse{4.22e-8^-2}\pgfmathresult
\end{codeexample}

    The choice |fixed| returns normal fixed point numbers and provides the
    highest compatibility with the \pgfname\ engine. It is activated
    automatically in case the FPU scales results.
    %
\begin{codeexample}[preamble={\usepgflibrary{fpu}}]
\pgfkeys{/pgf/fpu,/pgf/fpu/output format=fixed}
\pgfmathparse{sqrt(1e-12)}\pgfmathresult
\end{codeexample}
    %
\end{key}

\begin{key}{/pgf/fpu/scale results=\marg{scale}}
    A feature which allows semi-automatic result scaling. Setting this key has
    two effects: first, the output format for \emph{any} computation will be
    set to |fixed| (assuming results will be processed by \pgfname's kernel).
    Second, any expression which starts with a star, |*|, will be multiplied
    with \meta{scale}.
\end{key}

\begin{keylist}{
    /pgf/fpu/scale file plot x=\marg{scale},%
    /pgf/fpu/scale file plot y=\marg{scale},%
    /pgf/fpu/scale file plot z=\marg{scale}%
}
    These keys will patch \pgfname's |plot file| command to automatically scale
    single coordinates by \meta{scale}.

    The initial setting does not scale |plot file|.
\end{keylist}

\begin{command}{\pgflibraryfpuifactive\marg{true-code}\marg{false-code}}
    This command can be used to execute either \meta{true-code} or
    \meta{false-code}, depending on whether the FPU has been activated or not.
\end{command}


\subsection{Comparison to the fixed point arithmetics library}

There are other ways to increase the data range and/or the precision of
\pgfname's math parser. One of them is the |fp| package, preferable combined
with \pgfname's |fixedpointarithmetic| library. The differences between the FPU
and |fp| are:
%
\begin{itemize}
    \item The FPU supports at least the complete IEEE double precision number
        range, while |fp| covers only numbers of magnitude
        $\pm\pgfmathprintnumber{1e17}$.
    \item The FPU has a uniform relative precision of about 4--5 correct
        digits. The fixed point library has an absolute precision which may
        perform good in many cases -- but will fail at the ends of the data
        range (as every fixed point routines does).
    \item The FPU has potential to be faster than |fp| as it has access to fast
        mantissa operations using \pgfname's math capabilities (which use \TeX\
        registers).
\end{itemize}


\subsection{Command Reference and Programmer's Manual}

\subsubsection{Creating and Converting Floats}

\begin{command}{\pgfmathfloatparsenumber\marg{x}}
    Reads a number of arbitrary magnitude and precision and stores its result
    into |\pgfmathresult| as floating point number $m \cdot 10^e$ with mantissa
    and exponent base~$10$.

    The algorithm and the storage format is purely text-based. The number is
    stored as a triple of flags, a positive mantissa and an exponent, such as
    %
\begin{codeexample}[]
\pgfmathfloatparsenumber{2}
\pgfmathresult
\end{codeexample}
    %
    Please do not rely on the low-level representation here, use
    |\pgfmathfloattomacro| (and its variants) and |\pgfmathfloatcreate| if you
    want to work with these components.

    The flags encoded in |\pgfmathresult| are represented as a digit where
    `$0$' stands for the number $\pm 0\cdot 10^0$, `$1$' stands for a positive
    sign, `$2$' means a negative sign, `$3$' stands for `not a number', `$4$'
    means $+\infty$ and `$5$' stands for $-\infty$.

    The mantissa is a normalized real number $m \in \mathbb{R}$, $1 \le m <
    10$. It always contains a period and at least one digit after the period.
    The exponent is an integer.

    Examples:
    %
\begin{codeexample}[]
\pgfmathfloatparsenumber{0}
\pgfmathfloattomacro{\pgfmathresult}{\F}{\M}{\E}
Flags: \F; Mantissa \M; Exponent \E.
\end{codeexample}

\begin{codeexample}[]
\pgfmathfloatparsenumber{0.2}
\pgfmathfloattomacro{\pgfmathresult}{\F}{\M}{\E}
Flags: \F; Mantissa \M; Exponent \E.
\end{codeexample}

\begin{codeexample}[]
\pgfmathfloatparsenumber{42}
\pgfmathfloattomacro{\pgfmathresult}{\F}{\M}{\E}
Flags: \F; Mantissa \M; Exponent \E.
\end{codeexample}

\begin{codeexample}[]
\pgfmathfloatparsenumber{20.5E+2}
\pgfmathfloattomacro{\pgfmathresult}{\F}{\M}{\E}
Flags: \F; Mantissa \M; Exponent \E.
\end{codeexample}

\begin{codeexample}[]
\pgfmathfloatparsenumber{1e6}
\pgfmathfloattomacro{\pgfmathresult}{\F}{\M}{\E}
Flags: \F; Mantissa \M; Exponent \E.
\end{codeexample}

\begin{codeexample}[]
\pgfmathfloatparsenumber{5.21513e-11}
\pgfmathfloattomacro{\pgfmathresult}{\F}{\M}{\E}
Flags: \F; Mantissa \M; Exponent \E.
\end{codeexample}
    %
    The argument \meta{x} may be given in fixed point format or the scientific
    ``e'' (or ``E'') notation. The scientific notation does not necessarily
    need to be normalized. The supported exponent range is (currently) only
    limited by the \TeX-integer range (which uses 31 bit integer numbers).
\end{command}

\begin{key}{/pgf/fpu/handlers/empty number=\marg{input}\marg{unreadable part}}
    This command key is invoked in case an empty string is parsed inside of
    |\pgfmathfloatparsenumber|. You can overwrite it to assign a replacement
    |\pgfmathresult| (in float!).

    The initial setting is to invoke |invalid number|, see below.
\end{key}

\begin{key}{/pgf/fpu/handlers/invalid number=\marg{input}\marg{unreadable part}}
    This command key is invoked in case an invalid string is parsed inside of
    |\pgfmathfloatparsenumber|. You can overwrite it to assign a replacement
    |\pgfmathresult| (in float!).

    The initial setting is to generate an error message.
\end{key}

\begin{key}{/pgf/fpu/handlers/wrong lowlevel format=\marg{input}\marg{unreadable part}}
    This command key is invoked whenever |\pgfmathfloattoregisters| or its
    variants encounter something which is not a properly formatted low-level
    floating point number. As for |invalid number|, this key may assign a new
    |\pgfmathresult| (in floating point) which will be used instead of the
    offending \meta{input}.

    The initial setting is to generate an error message.
\end{key}

\begin{command}{\pgfmathfloatqparsenumber\marg{x}}
    The same as |\pgfmathfloatparsenumber|, but does not perform sanity checking.
\end{command}

\begin{command}{\pgfmathfloattofixed{\marg{x}}}
    Converts a number in floating point representation to a fixed point number.
    It is a counterpart to |\pgfmathfloatparsenumber|. The algorithm is purely
    text based and defines |\pgfmathresult| as a string sequence which
    represents the floating point number \meta{x} as a fixed point number (of
    arbitrary precision).
    %
\begin{codeexample}[]
\pgfmathfloatparsenumber{0.00052}
\pgfmathfloattomacro{\pgfmathresult}{\F}{\M}{\E}
Flags: \F; Mantissa \M; Exponent \E
$\to$
\pgfmathfloattofixed{\pgfmathresult}
\pgfmathresult
\end{codeexample}

\begin{codeexample}[]
\pgfmathfloatparsenumber{123.456e4}
\pgfmathfloattomacro{\pgfmathresult}{\F}{\M}{\E}
Flags: \F; Mantissa \M; Exponent \E
$\to$
\pgfmathfloattofixed{\pgfmathresult}
\pgfmathresult
\end{codeexample}
    %
\end{command}

\begin{command}{\pgfmathfloattoint\marg{x}}
    Converts a number from low-level floating point representation to an
    integer (by truncating the fractional part).
    %
\begin{codeexample}[]
\pgfmathfloatparsenumber{123456}
\pgfmathfloattoint{\pgfmathresult}
\pgfmathresult
\end{codeexample}

    See also |\pgfmathfloatint| which returns the result as float.
\end{command}

\begin{command}{\pgfmathfloattosci\marg{float}}
    Converts a number from low-level floating point representation to
    scientific format, $1.234e4$. The result will be assigned to the macro
    |\pgfmathresult|.
\end{command}

\begin{command}{\pgfmathfloatvalueof\marg{float}}
    Expands a number from low-level floating point representation to scientific
    format, $1.234e4$.

    Use |\pgfmathfloatvalueof| in contexts where only expandable macros are
    allowed.
\end{command}

\begin{command}{\pgfmathfloatcreate{\marg{flags}}{\marg{mantissa}}{\marg{exponent}}}
    Defines |\pgfmathresult| as the floating point number encoded by
    \meta{flags}, \meta{mantissa} and \meta{exponent}.

    All arguments are characters and will be expanded using |\edef|.
    %
\begin{codeexample}[]
\pgfmathfloatcreate{1}{1.0}{327}
\pgfmathfloattomacro{\pgfmathresult}{\F}{\M}{\E}
Flags: \F; Mantissa \M; Exponent \E
\end{codeexample}
    %
\end{command}

\begin{command}{\pgfmathfloatifflags\marg{floating point number}\marg{flag}\marg{true-code}\marg{false-code}}
    Invokes \meta{true-code} if the flag of \meta{floating point number} equals
    \meta{flag} and \meta{false-code} otherwise.

    The argument \meta{flag} can be one of
    %
    \begin{description}
        \item[0] to test for zero,
        \item[1] to test for positive numbers,
        \item[+] to test for positive numbers,
        \item[2] to test for negative numbers,
        \item[-] to test for negative numbers,
        \item[3] for ``not-a-number'',
        \item[4] for $+\infty$,
        \item[5] for $-\infty$.
    \end{description}
    %
\begin{codeexample}[preamble={\usetikzlibrary{fpu}}]
\pgfmathfloatparsenumber{42}
\pgfmathfloatifflags{\pgfmathresult}{0}{It's zero!}{It's not zero!}
\pgfmathfloatifflags{\pgfmathresult}{1}{It's positive!}{It's not positive!}
\pgfmathfloatifflags{\pgfmathresult}{2}{It's negative!}{It's not negative!}

% or, equivalently
\pgfmathfloatifflags{\pgfmathresult}{+}{It's positive!}{It's not positive!}
\pgfmathfloatifflags{\pgfmathresult}{-}{It's negative!}{It's not negative!}
\end{codeexample}
    %
\end{command}

\begin{command}{\pgfmathfloattomacro{\marg{x}}{\marg{flagsmacro}}{\marg{mantissamacro}}{\marg{exponentmacro}}}
    Extracts the flags of a floating point number \meta{x} to
    \meta{flagsmacro}, the mantissa to \meta{mantissamacro} and the exponent to
    \meta{exponentmacro}.
\end{command}

\begin{command}{\pgfmathfloattoregisters{\marg{x}}{\marg{flagscount}}{\marg{mantissadimen}}{\marg{exponentcount}}}
    Takes a floating point number \meta{x} as input and writes flags to count
    register \meta{flagscount}, mantissa to dimen register \meta{mantissadimen}
    and exponent to count register \meta{exponentcount}.

    Please note that this method rounds the mantissa to \TeX-precision.
\end{command}

\begin{command}{\pgfmathfloattoregisterstok{\marg{x}}{\marg{flagscount}}{\marg{mantissatoks}}{\marg{exponentcount}}}
    A variant of |\pgfmathfloattoregisters| which writes the mantissa into a
    token register. It maintains the full input precision.
\end{command}

\begin{command}{\pgfmathfloatgetflags{\marg{x}}{\marg{flagscount}}}
    Extracts the flags of \meta{x} into the count register \meta{flagscount}.
\end{command}

\begin{command}{\pgfmathfloatgetflagstomacro{\marg{x}}{\marg{macro}}}
    Extracts the flags of \meta{x} into the macro \meta{macro}.
\end{command}

\begin{command}{\pgfmathfloatgetmantissa{\marg{x}}{\marg{mantissadimen}}}
    Extracts the mantissa of \meta{x} into the dimen register
    \meta{mantissadimen}.
\end{command}

\begin{command}{\pgfmathfloatgetmantissatok{\marg{x}}{\marg{mantissatoks}}}
    Extracts the mantissa of \meta{x} into the token register
    \meta{mantissatoks}.
\end{command}

\begin{command}{\pgfmathfloatgetexponent{\marg{x}}{\marg{exponentcount}}}
    Extracts the exponent of \meta{x} into the count register
    \meta{exponentcount}.
\end{command}


\subsubsection{Symbolic Rounding Operations}

Commands in this section constitute the basic level implementations of the
rounding routines. They work symbolically, i.e.\ they operate on text, not on
numbers and allow arbitrarily large numbers.

\begin{command}{\pgfmathroundto{\marg{x}}}
    Rounds a fixed point number to prescribed precision and writes the result
    to |\pgfmathresult|.

    The desired precision can be configured with
    |/pgf/number format/precision|, see section~\ref{pgfmath-numberprinting}.
    This section does also contain application examples.

    Any trailing zeros after the period are discarded. The algorithm is purely
    text based and allows to deal with precisions beyond \TeX's fixed point
    support.

    As a side effect, the global boolean |\ifpgfmathfloatroundhasperiod| will
    be set to true if and only if the resulting mantissa has a period.
    Furthermore, |\ifpgfmathfloatroundmayneedrenormalize| will be set to true
    if and only if the rounding result's floating point representation would
    have a larger exponent than \meta{x}.
    %
\begin{codeexample}[]
\pgfmathroundto{1}
\pgfmathresult
\end{codeexample}
    %
\begin{codeexample}[]
\pgfmathroundto{4.685}
\pgfmathresult
\end{codeexample}
    %
\begin{codeexample}[]
\pgfmathroundto{19999.9996}
\pgfmathresult
\end{codeexample}
    %
\end{command}

\begin{command}{\pgfmathroundtozerofill{\marg{x}}}
    A variant of |\pgfmathroundto| which always uses a fixed number of digits
    behind the period. It fills missing digits with zeros.
    %
\begin{codeexample}[]
\pgfmathroundtozerofill{1}
\pgfmathresult
\end{codeexample}
    %
\begin{codeexample}[]
\pgfmathroundto{4.685}
\pgfmathresult
\end{codeexample}
    %
\begin{codeexample}[]
\pgfmathroundtozerofill{19999.9996}
\pgfmathresult
\end{codeexample}
    %
\end{command}

\begin{command}{\pgfmathfloatround{\marg{x}}}
    Rounds a normalized floating point number to a prescribed precision and
    writes the result to |\pgfmathresult|.

    The desired precision can be configured with
    |/pgf/number format/precision|, see section~\ref{pgfmath-numberprinting}.

    This method employs |\pgfmathroundto| to round the mantissa and applies
    renormalization if necessary.

    As a side effect, the global boolean |\ifpgfmathfloatroundhasperiod| will
    be set to true if and only if the resulting mantissa has a period.
    %
\begin{codeexample}[]
\pgfmathfloatparsenumber{52.5864}
\pgfmathfloatround{\pgfmathresult}
\pgfmathfloattosci{\pgfmathresult}
\pgfmathresult
\end{codeexample}
    %
\begin{codeexample}[]
\pgfmathfloatparsenumber{9.995}
\pgfmathfloatround{\pgfmathresult}
\pgfmathfloattosci{\pgfmathresult}
\pgfmathresult
\end{codeexample}
    %
\end{command}

\begin{command}{\pgfmathfloatroundzerofill{\marg{x}}}
    A variant of |\pgfmathfloatround| produces always the same number of digits
    after the period (it includes zeros if necessary).
    %
\begin{codeexample}[]
\pgfmathfloatparsenumber{52.5864}
\pgfmathfloatroundzerofill{\pgfmathresult}
\pgfmathfloattosci{\pgfmathresult}
\pgfmathresult
\end{codeexample}
    %
\begin{codeexample}[]
\pgfmathfloatparsenumber{9.995}
\pgfmathfloatroundzerofill{\pgfmathresult}
\pgfmathfloattosci{\pgfmathresult}
\pgfmathresult
\end{codeexample}
    %
\end{command}


\subsubsection{Math Operations Commands}

This section describes some of the replacement commands in more detail.

Please note that these commands can be used even if the |fpu| as such has not
been activated -- it is sufficient to load the library.

\begin{command}{\pgfmathfloat\meta{op}}
    Methods of this form constitute the replacement operations where \meta{op}
    can be any of the well-known math operations.

    Thus, \declareandlabel{\pgfmathfloatadd} is the counterpart for
    |\pgfmathadd| and so on. The semantics and number of arguments is the same,
    but all input and output arguments are \emph{expected} to be floating point
    numbers.
\end{command}

\begin{command}{\pgfmathfloattoextentedprecision{\marg{x}}}
    Renormalizes \meta{x} to extended precision mantissa, meaning $100 \le m <
    1000$ instead of $1 \le m < 10$.

    The ``extended precision'' means we have higher accuracy when we apply
    pgfmath operations to mantissas.

    The input argument is expected to be a normalized floating point number;
    the output argument is a non-normalized floating point number (well,
    normalized to extended precision).

    The operation is supposed to be very fast.
\end{command}

\begin{command}{\pgfmathfloatsetextprecision\marg{shift}}
    Sets the precision used inside of |\pgfmathfloattoextentedprecision| to
    \meta{shift}.

    The different choices are

    \begin{tabular}{llrll}
        0 & normalization to &    $0$ & $\le m < 1$     & (disable extended precision)                    \\
        1 & normalization to &   $10$ & $\le m < 100$   &                                                 \\
        2 & normalization to &  $100$ & $\le m < 1000$  & (default of |\pgfmathfloattoextentedprecision|) \\
        3 & normalization to & $1000$ & $\le m < 10000$ &                                                 \\
    \end{tabular}
\end{command}

\begin{command}{\pgfmathfloatlessthan{\marg{x}}{\marg{y}}}
    Defines |\pgfmathresult| as $1.0$ if $\meta{x} < \meta{y}$, but $0.0$
    otherwise. It also sets the global \TeX-boolean |\pgfmathfloatcomparison|
    accordingly. The arguments \meta{x} and \meta{y} are expected to be numbers
    which have already been processed by |\pgfmathfloatparsenumber|. Arithmetic
    is carried out using \TeX-registers for exponent- and mantissa comparison.
\end{command}

\begin{command}{\pgfmathfloatmultiplyfixed\marg{float}\marg{fixed}}
    Defines |\pgfmathresult| to be $\meta{float} \cdot \meta{fixed}$ where
    \meta{float} is a floating point number and \meta{fixed} is a fixed point
    number. The computation is performed in floating point arithmetics, that
    means we compute $m \cdot \meta{fixed}$ and renormalize the result where
    $m$ is the mantissa of \meta{float}.

    This operation renormalizes \meta{float} with
    |\pgfmathfloattoextentedprecision| before the operation, that means it is
    intended for relatively small arguments of \meta{fixed}. The result is a
    floating point number.
\end{command}

\begin{command}{\pgfmathfloatifapproxequalrel\marg{a}\marg{b}\marg{true-code}\marg{false-code}}
    Computes the relative error between \meta{a} and \meta{b} (assuming
    \meta{b}$\neq 0$) and invokes \meta{true-code} if the relative error is
    below |/pgf/fpu/rel thresh| and \meta{false-code} if that is not the case.

    The input arguments will be parsed with |\pgfmathfloatparsenumber|.

    \begin{key}{/pgf/fpu/rel thresh=\marg{number} (initially 1e-4)}
        A threshold used by |\pgfmathfloatifapproxequalrel| to decide whether
        numbers are approximately equal.
    \end{key}
\end{command}

\begin{command}{\pgfmathfloatshift{\marg{x}}{\marg{num}}}
    Defines |\pgfmathresult| to be $\meta{x} \cdot 10^{\meta{num}}$. The
    operation is an arithmetic shift base ten and modifies only the exponent of
    \meta{x}. The argument \meta{num} is expected to be a (positive or
    negative) integer.
\end{command}

\begin{command}{\pgfmathfloatabserror\marg{x}\marg{y}}
    Defines |\pgfmathresult| to be the absolute error between two floating
    point numbers $x$ and $y$, $\lvert x - y\rvert $ and returns the result as
    floating point number.
\end{command}

\begin{command}{\pgfmathfloatrelerror\marg{x}\marg{y}}
    Defines |\pgfmathresult| to be the relative error between two floating
    point numbers $x$ and $y$, $\lvert x - y\rvert / \lvert y \rvert$ and
    returns the result as floating point number.
\end{command}

\begin{command}{\pgfmathfloatint\marg{x}}
    Returns the integer part of the floating point number \meta{x}, by
    truncating any digits after the period. This methods truncates the absolute
    value $\rvert x \lvert$ to the next smaller integer and restores the
    original sign afterwards.

    The result is returned as floating point number as well.

    See also |\pgfmathfloattoint| which returns the number in integer format.
\end{command}

\begin{command}{\pgfmathlog{\marg{x}}}
    Defines |\pgfmathresult| to be the natural logarithm of \meta{x},
    $\ln(\meta{x})$. This method is logically the same as |\pgfmathln|, but it
    applies floating point arithmetics to read number \meta{x} and employs the
    logarithm identity \[ \ln(m \cdot 10^e) = \ln(m) + e \cdot \ln(10) \] to
    get the result. The factor $\ln(10)$ is a constant, so only $\ln(m)$ with
    $1 \le m < 10$ needs to be computed. This is done using standard pgf math
    operations.

    Please note that \meta{x} needs to be a number, expression parsing is not
    possible here.

    If \meta{x} is \emph{not} a bounded positive real number (for example
    $\meta{x} \le 0$), |\pgfmathresult| will be \emph{empty}, no error message
    will be generated.
    %
\begin{codeexample}[preamble={\usetikzlibrary{fpu}}]
\pgfmathlog{1.452e-7}
\pgfmathresult
\end{codeexample}
    %
\begin{codeexample}[preamble={\usetikzlibrary{fpu}}]
\pgfmathlog{6.426e+8}
\pgfmathresult
\end{codeexample}
    %
\end{command}


\subsubsection{Accessing the Original Math Routines for Programmers}

As soon as the library is loaded, every private math routine will be copied to
a new name. This allows library and package authors to access the \TeX-register
based math routines even if the FPU is activated. And, of course, it allows the
FPU as such to perform its own mantissa computations.

The private implementations of \pgfname\ math commands, which are of the form
|\pgfmath|\meta{name}|@|, will be available as|\pgfmath@basic@|\meta{name}|@|
as soon as the library is loaded.

\endgroup
