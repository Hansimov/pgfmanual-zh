% \part{Ti\emph{k}Z ist \emph{kein} Zeichenprogramm}
\part{绘何物为}\footnote{\color{blue} 该部分标题原文为德文:Ti\emph{k}Z ist \emph{kein} Zeichenprogramm,翻译成英文就是 “Ti\emph{k}Z is not a drawing program”,中文意思是“Ti\emph{k}Z 不是一个绘图程序”。然而德文采用的是 “GNU's Not Unix!” 式的递归缩写,这里如果直接采用原文也并非不可以,但是中文博大精深,一定有对应的贴切的翻译。
\\ \indent \indent 我这里译成“绘何物为”,用了拼音的递归:Huì hé wù wéi。意即“‘绘’是什么呢”,当然也可以将“绘”直接作为动词,理解成“绘制什么呢”。这样中文含义就和原文含义形成一问一答,无论是形式上还是内容上,都有了合理的对应。当然,这里着实夹杂了译者的私货,正文中依旧使用 \tikzname\ 来指代这一绘图系统。}
\label{part-tikz}

{\Large \emph{by Till Tantau}}


\bigskip
\noindent
\vskip3cm
\begin{codeexample}[graphic=white]

\begin{tikzpicture}[angle radius=.75cm]

  \node (A) at (-2,0)    [red,left]   {$A$};
  \node (B) at ( 3,.5)    [red,right]  {$B$};
  \node (C) at (-2,2)  [blue,left]  {$C$};
  \node (D) at ( 3,2.5)  [blue,right] {$D$};
  \node (E) at (60:-5mm) [below]      {$E$};
  \node (F) at (60:3.5cm)  [above]      {$F$};

  \coordinate (X) at (intersection cs:first line={(A)--(B)}, second line={(E)--(F)});
  \coordinate (Y) at (intersection cs:first line={(C)--(D)}, second line={(E)--(F)});
  
  \path 
    (A) edge [red, thick]  (B)
    (C) edge [blue, thick] (D)
    (E) edge [thick]       (F)
      pic ["$\alpha$",  draw, fill=yellow]   {angle = F--X--A}
      pic ["$\beta$",   draw, fill=green!30] {angle = B--X--F}
      pic ["$\gamma$",  draw, fill=yellow]   {angle = E--Y--D}
      pic ["$\delta$",  draw, fill=green!30] {angle = C--Y--E};

  \node at ($ (D)!.5!(B) $) [right=1cm,text width=6cm,rounded corners,fill=red!20,inner sep=1ex]
    {
      When we assume that $\color{red}AB$ and $\color{blue}CD$ are
      parallel, i.\,e., ${\color{red}AB} \mathbin{\|} \color{blue}CD$,
      then $\alpha = \delta$ and $\beta = \gamma$.
    };
\end{tikzpicture}
\end{codeexample}

\include{./text-zh/pgfmanual-zh-tikz/pgfmanual-zh-tikz-design}
% \include{./text-zh/pgfmanual-zh-tikz/pgfmanual-zh-tikz-scopes}
% \include{./text-zh/pgfmanual-zh-tikz/pgfmanual-zh-tikz-coordinates}
% \include{./text-zh/pgfmanual-zh-tikz/pgfmanual-zh-tikz-paths}
% \include{./text-zh/pgfmanual-zh-tikz/pgfmanual-zh-tikz-actions}
% \include{./text-zh/pgfmanual-zh-tikz/pgfmanual-zh-tikz-arrows}
% \include{./text-zh/pgfmanual-zh-tikz/pgfmanual-zh-tikz-shapes}
% \include{./text-zh/pgfmanual-zh-tikz/pgfmanual-zh-tikz-pics}
% \include{./text-zh/pgfmanual-zh-tikz/pgfmanual-zh-tikz-graphs}
% \include{./text-zh/pgfmanual-zh-tikz/pgfmanual-zh-tikz-matrices}
% \include{./text-zh/pgfmanual-zh-tikz/pgfmanual-zh-tikz-trees}
% \include{./text-zh/pgfmanual-zh-tikz/pgfmanual-zh-tikz-plots}
% \include{./text-zh/pgfmanual-zh-tikz/pgfmanual-zh-tikz-transparency}
% \include{./text-zh/pgfmanual-zh-tikz/pgfmanual-zh-tikz-decorations}
% \include{./text-zh/pgfmanual-zh-tikz/pgfmanual-zh-tikz-transformations}